\section{Related Work}
\begin{frame}
\frametitle{Modern Portfolio Theory (MPT)}
\begin{columns}
\begin{column}{0.55\textwidth}
Produce a portfolio from the prediction of future performance with maximum expected return from given risk or minimum risk from given expected return
\begin{center}
\includegraphics[width=5cm]{images/efficient_frontier.png}
\end{center}
\end{column}
\begin{column}{0.45\textwidth}
\begin{center}
\includegraphics[width=4.8cm]{images/mpt.png}
\end{center}
\end{column}
\end{columns}
\end{frame}



\begin{frame}{Reinforcement Learning with Sharpe Ratio}
Sharpe Ratio ratio is defined as 
\[ SR = \frac{E(R_a - R_b)}{\sigma_a},
\sigma_a = \sqrt{VAR(R_a-R_b)}\]
where \(R_a\) is the return of the assert, 
\(R_b\) is the risk-free return,
\(E(R_a - R_b)\) is the expected excess return of the assert,
and \(\sigma_a\) is standard deviation of the excess return.
\end{frame}
\begin{frame}{The differential Sharpe ratio}
The differential Sharpe ratio \(D\) is defined in the form:
\[
\cfrac{d D_t}{d R_t} = 
\cfrac{B_{t-1}-A_{t-1} R_t}{(B_{t-1}-A_{t-1}^2)^\frac{3}{2}}
\]
where
A and B is the first and second moments of the returns' distributions
\[ A_n = \cfrac{1}{n}\sum_{i=1}^nR_i\quad
B_n = \cfrac{1}{n}\sum_{i=1}^nR_i^2
\]
\end{frame}
\begin{frame}{Reinforcement Learning with Sterling Ratio}
\[
Sterling Ratio=\frac{Annualized Average Return}{Maximum Drawdown}
\]
Maximum Drawdown is cumbersome to minimize, used DD instead, the square root of the average of the
square of the negative returns, defined as
\[
DD_T = \sqrt{\cfrac{1}{T}\sum_{t=0}^{T}{min\{R_T,0\}^2}}
\]
Sterling Ratio than can be replaced by downside deviation ratio (DDR)
\[
DDR_T = \frac{Average(R_T)}{DD_T}
\]
\end{frame}



\begin{frame}{Differential form}
The reward function will use the differential form of the utility function \(D_t\).
\[
D_t = 
\begin{cases}
    \cfrac{R_{t-1} -\frac{1}{2}A_{t-1}}{DD_{t-1}},&\text{if  }R_t > 0\\
    \cfrac{DD_{t-1}^2 (R_{t-1}-\frac{1}{2}A_{t-1})  -\frac{1}{2}A_{t-1} R_t^2}{DD_{t-1}^3},&\text{if  }R_t \leq 0
\end{cases}
\]
Unlike utility functions that use variance as the risk-adjusted factor, this formula indicates no penalty for large positive returns. 
\end{frame}